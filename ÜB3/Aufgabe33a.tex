\documentclass{scrartcl}
\usepackage[utf8]{inputenc} % Umlaute werden erkannt
\usepackage{graphicx} % um png-Dateien einzubinden

\title{Aufgabe 3.3}
\date{}

\begin{document}
\maketitle

\section{}

\begin{figure}[ht]
	\centering
	\includegraphics[width=500pt]{Aufgabe33a.png}
\end{figure}

\section{}

Die Strategie $RR_{1}$ belegt die CPU maximal oft, d.h. es gibt nur minimalen
Leerlauf - in unserem Beispiel sogar überhaupt keinen. Leerlauf kann es nur
geben, wenn kein Prozess ausführbereit ist. In diesem Fall wäre aber auch keine
andere Strategie schneller. Also ist $RR_{1}$ die beste Strategie.



\end{document}
