\documentclass{article}
\usepackage[utf8]{inputenc}
\usepackage{amsmath}
\usepackage{amssymb}
\usepackage[left=2cm,right=2cm,bottom=2cm,top=2cm]{geometry}
\usepackage{tikz}

\title{Bus - Blatt 6}
\date{}

\begin{document}
\maketitle

\section*{Aufgabe 4}

\subsection*{a)}

Die Seitengröße ist $32 KiB = 2^{10}*32 Bit = 2^{10+5} Bit = 2^{15} Bit$.
Außerdem sind die Adressen 48 Bit lang, also können wir $2^{48}/2^{15} = 2^{33}$
Seiten speichern und die Page Table hat $2^{33}$ Einträge.

\subsection*{b)}

Bei einer einstufigen Page Table hätten wir $2^{33}$ Einträge. Wenn wir nun
eine dreistufige Page Table mit jeweils gleicher Größe betrachten, dann gilt:\\
Angenommen die Page Tables haben jeweils $m$ Einträge. Dann können wir
im ersten Level $m$ Seiten ansprechen, im zweiten Level $m^2$ und im dritten
Level $m^3$. Wir müssen $2^{33}$ Level ansprechen und es gilt
$m^3 = 2^33 \Leftrightarrow m = 2^{11}$, also haben die Tabellen jeweils
$ 2^{11} $ Einträge und insgesamt sind es $ 3 * 2^{11} $ Einträge.

\subsection*{c)}

Durch mehrstufiges Paging werden die Page Tables deutlich kleiner, wie man
auch sehen kann, wenn man das Ergebnis von Aufgabe a) und b) vergleicht.
Allerdings wird die Zugriffszeit mit jeder weiteren Stufe um eines größer:\\
Bei zweistufigem Paging ist sie doppelt so groß wie bei einstufigem Paging
und bei dreistufigem Paging schon dreimal so groß.

\end{document}
