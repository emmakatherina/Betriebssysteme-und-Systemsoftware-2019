\documentclass{scrartcl}
\usepackage[utf8]{inputenc} % Umlaute werden erkannt
\usepackage{graphicx} % um png-Dateien einzubinden

\title{Übungsblatt 2}
\subtitle{Betriebssysteme und Systemsoftware, Sommersemester 2019}
\author{ Hanbit Chang, 374370 \and Jonghwan Choi, 365634 \and Emma Ahrens, 371063}

\begin{document}
\maketitle

% Aufgabe 1
\section{Aufgabe}

Siehe Programmdateien.

% Aufgabe 2
\section{Aufgabe}

\subsection{}
Mit top kann man sich die Liste aller laufenden Prozesse anzeigen lassen. In der Liste sucht man sich die eindeutige PID heraus und beendet den jeweiligen Prozess mit kill X in der Kommandozeile, wo X die PID ist.

\subsection{}
Die Prozesse lassen sich mit ps -e anzeigen.

\subsection{}
Wenn man in der vierten Zeile char mit int ersetzt, dann terminiert das Programm.
Das liegt daran, dass ein unsigned char nur die Werte 0 bis 255 annehmen kann und wenn i = 255 gilt, wird ++i zu i = 0 ausgewertet. Ein unsigned int kann jedoch $2^{16}$ Werte annehmen, wobei $500 < 2^{16}$ gilt. Deshalb haben wir hier nicht das Problem eines Overflows. 

% Aufgabe 3
\section{Aufgabe}

\subsection{}
Siehe Figure 1.

\begin{figure}[ht]
	\centering
	\includegraphics[width=0.33\textwidth]{Prozessbaum.png}
	\caption{Baum zur Aufgabe 3.1}
\end{figure}

\subsection{}
Da das Programm nur in den ersten Block geht, wenn fork() 0 ist, wird es in dem Block in der if-Bedingung immer fork() == 0 mit true auswerten. Also wird "u" ausgegeben und "S" kann nie ausgegeben werden.

% Aufgabe 4
\section{Aufgabe}

\subsection{}

Siehe Programmdateien.

\subsection{}

die Funktion \texttt{fprintf()} ist ein library call wird gebuffert. Also wird die Datei in den Speicher zwischengespeichert. Um die Datei auszugeben braucht man die Funktion \texttt{fflush()}. Damit wird die Datei es ausgegeben. \\
Bei write wird \texttt{fflush()} nicht verwendet, weil es ist ein system call und es wird direkt in Betriebssystem gesendet und es wird direkt ausgegeben. \\
(sprich: wenn Sie mehr als nur einen Schreibaufruf zwischen \texttt{fflush()} aufrufen durchführen)
In dieser Situation wäre \texttt{fprintf()} effizienter, da es möglich ist in \texttt{fprintf()} größere und mehere Daten gebündelt zu senden.

\subsection{}

Wenn ein Prozess fertig ist, aber der Elternprozess noch nicht fertig ist, dann bleibt der Prozess in dem Zustand TASK\_ZOMBIE noch erhalten, damit der Elternprozess auf die Daten zugreifen kann. Wenn der Elternprozess \texttt{wait4()} aufruft, wird der Prozess gelöscht.

\subsection{}

Der Kindprozess speichert die PID des Elternprozesses, damit das System weiß, welcher Prozess nach dem Kindprozess abgearbeitet werden soll. Also damit der Baum der Prozesse sinnvoll weiter abgearbeitet werden kann.

\end{document}
